% Created 2019-01-02 Wed 22:18
\documentclass[11pt]{article}
\usepackage[utf8]{inputenc}
\usepackage[T1]{fontenc}
\usepackage{fixltx2e}
\usepackage{graphicx}
\usepackage{longtable}
\usepackage{float}
\usepackage{wrapfig}
\usepackage{rotating}
\usepackage[normalem]{ulem}
\usepackage{amsmath}
\usepackage{textcomp}
\usepackage{marvosym}
\usepackage{wasysym}
\usepackage{amssymb}
\usepackage{hyperref}
\tolerance=1000
\usepackage{parskip}
\setlength{\parskip}{20pt}
\usepackage{underscore}
\usepackage{breakurl}
\usepackage{url}
\PassOptionsToPackage{hyphens}{url}
\usepackage{hyperref}
\author{Heini Fagerlund}
\date{\today}
\title{Open source projects by Heini Fagerlund (@hfagerlund)}
\hypersetup{
  pdfkeywords={},
  pdfsubject={},
  pdfcreator={Emacs 24.5.1 (Org mode 8.2.10)}}
\begin{document}

\maketitle

\section{Projects by @hfagerlund}
\label{sec-1}
\begin{itemize}
\item \textbf{machine-learning-classifier-iris}
  \paragraph{}
  A comparison of machine learning algorithms for making predictions written in Python (v3.7) and using a built-in dataset included with scikit-learn \cite{machinelearning}. Data visualizations are also included.
\paragraph{}
  This repository contains new feature development continued from the Jupyter version \cite{jupyter}.
\item \textbf{mkdocs-docskimmer}
\paragraph{}
docSkimmer is an accessible, responsive, minimalist theme originally built for the static documentation site-generator MkDocs. Visit the \href{http://bitsof.bytesofdesign.com/mkdocs-docskimmer/}{documentation site}.
\paragraph{}
The docSkimmer theme has also been ported for Jekyll.
\item \textbf{jekyll-docskimmer-theme}
   \paragraph{}
  jekyll-docskimmer-theme is the official port of \href{https://github.com/hfagerlund/mkdocs-docskimmer}{docSkimmer theme} to Jekyll.
\item \textbf{strip\_anchors}
   \paragraph{}
   Source code for the custom strip\_anchors Twig extension. This repository shows how the filter can be used in a Symfony 4 project to transform HTML content into valid RSS content for a feed \cite{stripanchors}.
\item \textbf{elections-carousel-component}
   \paragraph{}
   This project is a component-based carousel presentation of election results data from a JSON API. Built using React and Webpack \cite{elections-carousel-component}.
\item \textbf{hfagerlund.github.io}
   \paragraph{}
   My GitHub Pages site which currently serves as a table of contents listing of my open-source projects on GitHub \cite{ghpages}.
\item \textbf{git-add-msg}
   \paragraph{}
   git-add-msg is a git extension that can be used to automatically update Trac tickets with comments without leaving the command-line interface \cite{git-add-msg}.
\item \textbf{blog}
   \paragraph{}
   This (mirror) repository includes modualar scripts for a page scroller and page event analytics tracking \cite{blog}.
\end{itemize}

\begin{sloppypar}
\bibliographystyle{plain}
\bibliography{opensource-bib}
\end{sloppypar}
% Emacs 24.5.1 (Org mode 8.2.10)
\end{document}