% Created 2023-11-19 Sun 16:19
% Intended LaTeX compiler: pdflatex
\documentclass[11pt]{article}
\usepackage[utf8]{inputenc}
\usepackage[T1]{fontenc}
\usepackage{graphicx}
\usepackage{grffile}
\usepackage{longtable}
\usepackage{wrapfig}
\usepackage{rotating}
\usepackage[normalem]{ulem}
\usepackage{amsmath}
\usepackage{textcomp}
\usepackage{amssymb}
\usepackage{capt-of}
\usepackage{hyperref}
\usepackage{parskip}
\setlength{\parskip}{20pt}
\usepackage{underscore}
\usepackage{breakurl}
\usepackage{url}
\PassOptionsToPackage{hyphens}{url}
\usepackage{hyperref}
\author{Heini Fagerlund}
\date{\today}
\title{Open source projects by Heini Fagerlund (@hfagerlund)}
\hypersetup{
 pdfauthor={Heini Fagerlund},
 pdftitle={Open source projects by Heini Fagerlund (@hfagerlund)},
 pdfkeywords={},
 pdfsubject={},
 pdfcreator={Emacs 26.3 (Org mode 9.1.9)}, 
 pdflang={English}}
\begin{document}

\maketitle

\section{Projects by @hfagerlund}
\label{sec:orge9eef1b}
\begin{itemize}
\item \textbf{quiz\_app}
\paragraph{}
A 'quiz' app with a custom GraphQL API backend (using Laravel 10), and Vue3 front-end. Built using Sail (docker-compose) running a MySQL database \cite{quiz-app}.
\item \textbf{erlang-experiments}
\paragraph{}
The groundwork for a Dockerized Erlang app \cite{erlang}.
\item \textbf{hellofromlua\_rock}
\paragraph{}
A custom Lua rock. \cite{luarock}. Usage is shown in the hello-lua demo \cite{lua}.
\item \textbf{machine-learning-classifier-iris}
\paragraph{}
A comparison of machine learning algorithms for making predictions written in Python (v3.7) and using a built-in dataset included with scikit-learn \cite{machinelearning}. Data visualizations are also included.
 \paragraph{}
This repository contains new feature development continued from the Jupyter version \cite{jupyter}.
\item \textbf{mkdocs-docskimmer}
\paragraph{}
docSkimmer is an accessible, responsive, minimalist theme originally built for the static documentation site-generator MkDocs \cite{mkdocs-docskimmer}. A live demo is available at \href{http://bitsof.bytesofdesign.com/mkdocs-docskimmer/}{documentation site}.
\paragraph{}
The docSkimmer theme has also been ported for Jekyll \cite{jekyll-docskimmer}.
\item \textbf{mkdocs-docstyler-plugin}
\paragraph{}
docstyler is a plugin for MkDocs that adds persistent, preferred
and/or alternate stylesheet links to custom MkDocs themes \cite{mkdocs-docstyler-plugin}.
\item \textbf{jekyll-docskimmer-theme}
 \paragraph{}
jekyll-docskimmer-theme is the official port of \href{https://github.com/hfagerlund/mkdocs-docskimmer}{docSkimmer theme} to Jekyll \cite{jekyll-docskimmer}.
\item \textbf{strip\_anchors}
\paragraph{}
Source code for the custom strip\_anchors Twig extension. This repository shows how the filter can be used in a Symfony 4 project to transform HTML content into valid RSS content for a feed \cite{stripanchors}.
\item \textbf{elections-carousel-component}
\paragraph{}
This project is a component-based carousel presentation of election results data from a JSON API. Built using React and Webpack \cite{elections-carousel-component}.
\item \textbf{elm-in-progress}
\paragraph{}
Experimentation in Elm v0.19.0.
 \cite{elm}.
\item \textbf{hfagerlund.github.io}
\paragraph{}
My GitHub Pages site which currently serves as a table of contents listing of my open-source projects on GitHub \cite{ghpages}.
\item \textbf{osprojects}
\paragraph{}
A catalog of GitHub open source projects by Heini Fagerlund (@hfagerlund) in \TeX{}, org, and PDF formats \cite{osprojects}. Created using Emacs.
\item \textbf{react-app-starter}
\paragraph{}
react-app-starter is a simple Dockerized React app skeleton to be used as a time-saving and easy to understand starter for new React projects \cite{react-app-starter}.
\paragraph{}
\href{https://docs.github.com/en/github/creating-cloning-and-archiving-repositories/creating-a-repository-on-github/creating-a-repository-from-a-template}{How to use this repo as a template}
\item \textbf{git-add-msg}
\paragraph{}
git-add-msg is a git extension that can be used to automatically update Trac tickets with comments without leaving the command-line interface \cite{git-add-msg}.
\item \textbf{blog}
\paragraph{}
This (mirror) repository includes modular scripts for a page scroller and page event analytics tracking \cite{blog}.
\end{itemize}

\begin{LaTeX}
\begin{sloppypar}
\bibliographystyle{plain}
\bibliography{opensource-bib}
\end{sloppypar}
\end{LaTeX}
Emacs 26.3 (Org mode 9.1.9)
\end{document}
